\subsubsection{Logic Outline}
\begin{table*}[!hbp]
	\begin{tabular}{p{0.15\textwidth}|p{0.25\textwidth}|p{0.25\textwidth}|p{0.25\textwidth}}
		\hline
		步骤 & 期望目标 & 目前算法实际达到的效果 & 主要差异\\
		\hline
		Region Grow 
		& 输入:一个由多个物体混在一起没有被label的区域(要求不能有一个物体部分被label而另一部分没有被label)
		输出:欠分割的结果—————将每一帧分为多个patch每一个patch由一个或多个物体组成 
		& 将每一帧中还没有被label的区域按照空间是否近邻分为多个patch 
		& 只要输入符合要求我认为在室内数据中是等价的
		\\
		\hline
		Clustering
		& 输入所有的patch, 输出每个patch所属类别,要求每一类所有的patch中最大的(所占点最多)物体都相同
		& 以patch最多的一帧的patch特征为初始用k-means来获取聚类中心,将每帧中特征空间中距离聚类中心最近的归为该类
		& 完全不等价,也没有满足需求\\
		\hline
		Joint Registration 
		& 输入同一类的patch,将它们按照类中patch共有的最大的物体配准到一起 
		& 输出一个模型与一组刚体变换,在存在noise和outlier的前提下,以高斯为先验最大化 后验概率————这些观察到的patch是由这个模型按照这组刚体变换生成的  
		& 并不等价,但是从结果来看我觉得是能够满足需求的 \\
		\hline
		Graph Cut
		& 将每一类的每个patch中除了主体物体之外的部分重新标记为没有被label的区域
		& 一个superpixel如果按照某个类的变换能够在每一帧都匹配得(如能量项中的取max就是为了表达需要在每一帧中都匹配得好才叫匹配得好)很好则label为这个物体,果都匹配得不好则重新被标记为没有label的区域(匹配得好与不好的门限的设置依然是不清晰的)
		& 因为噪声和形变的存在所以并不能够等价
	\end{tabular}
	\caption{Logic Outline v0.0} %表格的名称
	\label{tab:logic_outline00}
\end{table*}
\begin{table*}[!hbp]
	\begin{tabular}{p{0.15\textwidth}|p{0.75\textwidth}}
		\hline
		步骤 & 期望目标 \\
		\hline
		Region Grow&\tabincell{l}{
			输出:\\
			每一帧每个pixel有一个label值
			要求:\\
			1.在任意一帧中,属于同一个物体的pixel必须被赋予同一个label值。\\
			2.在任意一帧中,属于不同的物体的pixel也可以被赋予同一个label的值。\\(被赋予同一个label值的pixel被认为组成一个patch,\\则这项要求等价于一个patch可以是一个或多个物体的集合)\\
			3.在不同的帧之间,属于同一个物体的pixel不需要被赋予同一个label值。}\\
		\hline
		Clustering&\tabincell{l}{输出:\\
			每个patch赋予一个新的label的值,其中可以存在一个特殊的零label\\
			要求:\\
			1.在不同帧之间,同一个label(零label除外)的patch(称为同一个类的patch)必须包含\\同一个物体,且这个相同的物体在包含它的patch中占有的点的比例足够大(称为类内核心物体)\\
			2.允许原本多个patch被同时赋予零label}\\
		\hline
		Joint Registration&\tabincell{l}{
			输出:\\
			除了零label的patch以外,
			对于每一个patch得到一个刚体变换,对于每一类得到一个物体模型。\\
			要求:\\
			1.零label除外,每一类的patch按照所得到的刚体变换变换以后,\\这一类的类内核心物体将会在三维空间中对齐。}\\
		\hline
		Graph Cut&\tabincell{l}{输出:\\
			更新每个pixel的label,使得原本属于类内核心物体的pixel的label与原本类的label相同,\\
			而不属于类内核心物体的pixel的label被置为零label\\
			要求:\\
			1.允许部分属于类内核心物体的pixel被置为零label。\\
			2.不允许不属于类内核心物体的pixel保留原来该类的label。}\\
		\hline
		Region Grow&\tabincell{l}{输出:\\
			给每一个零label的pixel赋予一个非零的label值。\\
			要求:\\
			1.对于不属于上一步中类内核心物体的pixel要求符合第一次\\region grow的要求(属于新分出来的物体的pixel)。}\\
		\hline
		Clustering&\tabincell{l}{输出:\\
			每个patch赋予一个新的label的值,不存在零label
			要求:\\
			1.在不同帧之间,同一个label(零label除外)的patch(称为同一个类的patch)\\必须包含同一个物体,且这个相同的物体在包含它的\\patch中占有的点的比例足够大(称为类内核心物体)\\
			2.允许patch恢复为上一步聚类时的label值}\\
		\hline
	\end{tabular}
	\caption{Logic Outline v0.1} %表格的名称
	\label{tab:logic_outline01}
\end{table*}