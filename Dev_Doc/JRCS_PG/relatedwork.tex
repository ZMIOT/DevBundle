\section{Related Work}
\label{sec:rw}
In this section we explain how our work is related to the previous works and how we draw experience from these previous works.
\subsection{Point Set Registration with GMM Representation}
\label{subsec:gmmreg}
There are a series of work that uses gaussion mixture model as representation for point set to formulate the registration problem.
\cite{CPD} consider the registration of two point sets as a probability density estimation problem. They force the Gaussian mixture model centroids to move coherently as a group to preserve the topological structure of the point sets. Their method is appliable to both rigid registration and non-rigid registration. As we highlighted in Section \ref{sec:intro}, our problem is different from the non-rigid registration considered in \cite{CPD}, the point drift could be non-coherent in our probelm.\cite{GMM_PAMI} summarized the works for point set registration using Gaussian mixture models and present a unified framework for the rigid and nonrigid point set registration problem. These works select one of the point set as the ``model''. Unlike these works, \cite{Evangelidis2014} treats all the point sets as data: they are all realizations of a Gaussian mixture and the registration is cast into a clustering problem. The recent work of \cite{GOGMA} pushes the idea to the application on a large scale of data. 
Comparing to these works, our work is most related to \cite{Evangelidis2014}. Our formulation can be seen as an extention of the formulation of \cite{Evangelidis2014} to simultaneously handle joint registration and co-segmentation.
\subsection{Image segmentation and co-segmentation}
\label{subsec:coseg}
\cite{grabcut} is an influential work for interactive image segmentation. It uses two Gaussian mixture model, one for foreground and one for background. To initialize these two Gaussian mixture models, \cite{grabcut} let users place a rectangle that contain the foreground. Our design of interaction draw on the experientce from \cite{grabcut}. The difference is that our interaction is designed for 3D space and can handle multiple objects segmentation rather than foreground-background segmentation. \cite{Taniai_2016_CVPR} jointly recover cosegmentation and dense per-pixel correspondence in two images. Its co-segmentation is limited to foreground-background segmentation. Our work solve a similar problem for multiple 3D point sets. 

\subsection{Segmentation from Motion}
The idea that motion can be strong hint for segmentation is used in many works.\cite{Xu:2015:ACS:2816795.2818075} employs a robot to do proactive push and track the motion to learn object segmentation. \cite{unsupervisededge} exploit the motion in video and use the motion edge as training data to learn an edge detector for image. These methods lean on the motion that is continuous in time and can be tracked. Our method can handle motion that is not continuous in time.

\subsection{3D Object Recognition based on Correspondence Grouping}
By allowing interactively input layout, the joint registration and co-segmentation problem can be treated as a series of 3D object recognition problem in point sets. Our method should be seen as one of the correspondence grouping method. Comparing to the previous methods of \cite{hough} and \cite{LOF}, our method simutaneously solve the problem for multiple target models in multiple scenes.