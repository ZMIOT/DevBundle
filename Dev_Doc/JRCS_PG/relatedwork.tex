\section{Related Work}
\label{sec:rw}
In this section, we review a series of related work on point set processing and how we draw experiences from these methods. 

%\subsection{Point Set Registration with GMM Representation}
%\label{subsec:gmmreg}
\noindent{\textbf{Point set registration with GMM representation.}}
Gaussian mixture models are widely used for point set registration problems due to its general ability of representing point sets for both rigid and non-rigid registrations and its robustness against noise.
%
A comprehensive survey about point set registration approaches using Gaussian mixture models can be found in~\cite{GMM_PAMI}. 
They also present a unified framework for rigid and nonrigid registration problems. 
%
These methods select one of the point sets as the ``template model" and fit other point sets to this ``template model".
%
Myronenko and Song consider the registration of two point sets as a probability density estimation problem~\cite{CPD}.
They use GMM to represent the geometry and force the GMM centroids to move coherently as a group to preserve the topological structure of the point sets. 
This method is applicable to both rigid registration and non-rigid registration. 
%
%(As we highlighted in Sec.~\ref{sec:intro}, our problem is different from the non-rigid registration while the point drift is non-coherent in our problem.)
%
Unlike above approaches, \cite{Evangelidis2014} treats all point sets equally as the realizations of a Gaussian mixture model and the registration is cast into a clustering problem. 
A more recent method pushes this idea to the application on a large-scale dataset~\cite{GOGMA}. 
Comparing to these methods, our method can be seen as an extension of the formulation of \cite{Evangelidis2014} to simultaneously handle joint registration and co-segmentation. The difference between our method and non-rigid registration is that we can handle the non-coherent point drift by simply estimating independent transformation for each object.
% 

\noindent{\textbf{Image segmentation and co-segmentation}.}
%
Due to the challenges in automatic segmentation problems, many interactive methods have been proposed to leverage human interaction on high-level hints and computational abilities of computers.
%
An influential technique for interactive image segmentation is GrabCut~\cite{grabcut}. 
It uses two Gaussian mixture models, one for foreground and the other for background. 
To initialize these two Gaussian mixture models, a rectangle is manually placed to contain the foreground. 
Our design of user interaction draws on the experience from \cite{grabcut}. 
%
The difference is that our tool is designed for 3D space and handles multi-object segmentation rather than a binary segmentation. 
%
\cite{Taniai_2016_CVPR} jointly recovers co-segmentation and dense per-pixel correspondences in two images. 
Its co-segmentation is limited to foreground-background segmentation. 
Our work solves a similar problem for multiple 3D point sets. 
\cxj{There are many image co-segmentation papers. If you want to discuss it, you should discuss more. }

\noindent{\textbf{Segmentation from motion.}}
Object motion, as a strong hint for object segmentation, is widely used in many approaches.
\cite{Xu:2015:ACS:2816795.2818075} employs a robot to do proactive pushes and tracks the motion to learn object segmentation. 
\cite{unsupervisededge} exploits motions in a video and uses the motion edges as training data to learn an edge detector for images.
These methods lean on the motion that is continuous over time and can be tracked. 
In comparison, our method handles motion that is non-continuous over time.

\noindent{\textbf{3D object recognition based on correspondence grouping.}}
By allowing interactively input the scene layout, the joint registration and co-segmentation problem can be treated as a series of 3D object recognition problems in point sets. 
%
Our method should be classified as one of the correspondence grouping method. 
Comparing to previous methods that uses \cxj{use what?}\hsy{you added "that uses" here I'm not sure }\cite{hough,LOF}, our method simultaneously solves the problem for multiple target models in multiple scenes.
\cxj{add more references of object recognitions here?}
