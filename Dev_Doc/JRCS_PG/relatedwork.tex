\section{Related Work}
\label{sec:rw}
In this section we explain how our work is related to the previous work on point set processing and how we draw experience from these methods.
\cxj{Work is typically used, not works. You can say previous approaches. }


%\subsection{Point Set Registration with GMM Representation}
%\label{subsec:gmmreg}
\paragraph{Point Set Registration with GMM Representation.}
There is a series of \mdf{approaches} that use Gaussion \cxj{Gaussian} mixture model as \mdf{the} representation for the point set registration problem \cxj{due to what advantages of GMM?}.
%
\cite{CPD} considers the registration of two point sets as a probability density estimation problem. 
They force the Gaussian mixture model centroids to move coherently as a group to preserve the topological structure of the point sets. Their method is appliable \cxj{applicable?} to both rigid registration and non-rigid registration. 
%
As we highlighted in Sec.~\ref{sec:intro}, our problem is different from the non-rigid registration while the point drift is non-coherent in our \mdf{problem}.
%
\cite{GMM_PAMI} summarizes the \mdf{approaches} for point set registration using Gaussian mixture models and presents a unified framework for rigid and nonrigid registration problem\mdf{s}. 
%
These methods select one of the point sets as the ``model''. \cxj{Is it better to use "template" than ``model''?} 
%
Unlike these works, \cite{Evangelidis2014} treats all point sets equally.
They are all realizations of a Gaussian mixture model and the registration is cast into a clustering problem. 
The recent work of \cite{GOGMA} pushes the idea to the application on a large scale data. 
Comparing to these methods, our method can be seen as an extension of the formulation of \cite{Evangelidis2014} to simultaneously handle joint registration and co-segmentation. 
\cxj{It should not be a simple extension. You should highlight what is the most challeging part we solved compared with others. I would say: } \mdf{In comparison, we employ the same GMM representation of object models while we formulate the non-coherent point drift as .....  }



%\subsection{Image segmentation and co-segmentation}
%\label{subsec:coseg}
\paragraph{Image segmentation and co-segmentation}
\cite{grabcut} is an influential work for interactive image segmentation. It uses two Gaussian mixture models, one for foreground and the other for background. 
To initialize these two Gaussian mixture models, a rectangle is manually placed to contain the foreground. Our design of interaction draws on the \mdf{experience} from \cite{grabcut}. 
%
The difference is that our interaction is designed for 3D space and can handle multi-object segmentation rather than foreground-background segmentation. \cite{Taniai_2016_CVPR} jointly recovers co-segmentation and dense per-pixel correspondences in two images. 
Its co-segmentation is limited to foreground-background segmentation. Our work solves a similar problem for multiple 3D point sets. 
\cxj{There are many image co-segmentation papers. If you want to discuss it, you should discuss more. }

\paragraph{Segmentation from Motion.}
The idea that motion can be a strong hint for segmentation is used in many works.\cite{Xu:2015:ACS:2816795.2818075} employs a robot to do proactive push and track the motion to learn object segmentation. \cite{unsupervisededge} exploits the motion in \mdf{a} video and uses the motion edges as the training data to learn an edge detector for image \cxj{use 'an image' or 'images'}. 
These methods lean on the motion that is continuous in time and can be tracked. Our method can handle motion that is not continuous in time.

\paragraph{3D Object Recognition based on Correspondence Grouping.}
By allowing interactively input \mdf{the scene layout}, the joint registration and co-segmentation problem can be treated as a series of 3D object recognition problem in point sets. Our method should be classified as one of the correspondence grouping method. Comparing to previous methods \cxj{that uses ...}~\cite{hough,LOF}, our method simutaneously \mdf{simultaneously} solve\mdf{s} the problem for multiple target models in multiple scenes.
