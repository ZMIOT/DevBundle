%%%%%%%%%%%%%%%%%%%%%%% file template.tex %%%%%%%%%%%%%%%%%%%%%%%%%
%
% This is a general template file for the LaTeX package SVJour3
% for Springer journals.          Springer Heidelberg 2010/09/16
%
% Copy it to a new file with a new name and use it as the basis
% for your article. Delete % signs as needed.
%
% This template includes a few options for different layouts and
% content for various journals. Please consult a previous issue of
% your journal as needed.
%
%%%%%%%%%%%%%%%%%%%%%%%%%%%%%%%%%%%%%%%%%%%%%%%%%%%%%%%%%%%%%%%%%%%
%
% First comes an example EPS file -- just ignore it and
% proceed on the \documentclass line
% your LaTeX will extract the file if required
\begin{filecontents*}{example.eps}
%!PS-Adobe-3.0 EPSF-3.0
%%BoundingBox: 19 19 221 221
%%CreationDate: Mon Sep 29 1997
%%Creator: programmed by hand (JK)
%%EndComments
gsave
newpath
  20 20 moveto
  20 220 lineto
  220 220 lineto
  220 20 lineto
closepath
2 setlinewidth
gsave
  .4 setgray fill
grestore
stroke
grestore
\end{filecontents*}
%
\RequirePackage{fix-cm}
%
%\documentclass{svjour3}                     % onecolumn (standard format)
%\documentclass[smallcondensed]{svjour3}     % onecolumn (ditto)
%\documentclass[smallextended]{svjour3}       % onecolumn (second format)
\documentclass[twocolumn]{svjour3}          % twocolumn
%
\smartqed  % flush right qed marks, e.g. at end of proof
%
\usepackage{graphicx}
\usepackage{amsmath}
\usepackage{amsfonts}
\usepackage{algorithm}
%
\usepackage{mathptmx}      % use Times fonts if available on your TeX system
%\usepackage{cite}
%\usepackage{url}
\usepackage{hyperref}
% insert here the call for the packages your document requires
%\usepackage{latexsym}
% etc.
%
% please place your own definitions here and don't use \def but
% \newcommand{}{}
%\newcommand{\cxj}[1]{\textcolor[rgb]{1.00,0.00,0.00}{(cxj: #1)}}
%\newcommand{\mdf}[1]{\textcolor[rgb]{1.00,0.00,1.00}{#1}}
%\newcommand{\hsy}[1]{\textcolor[rgb]{0.44,0.67,0.22}{(hsy: #1)}}
%\newcommand{\confirm}[1]{\textcolor[rgb]{0.00,1.00,1.00}{#1}}
\newcommand{\vb}[1]{\mathbf{#1}}
\newcommand{\defV}{\mathcal{V}}
\newcommand{\defZ}{\mathcal{Z}}
\newcommand{\comments}[1]{}
%
% Insert the name of "your journal" with
% \journalname{myjournal}
%
\begin{document}

\title{Point Sets Joint Registration and Co-segmentation%\thanks{Grants or other notes
%about the article that should go on the front page should be
%placed here. General acknowledgments should be placed at the end of the article.}
}
\subtitle{}

%\titlerunning{Short form of title}        % if too long for running head

\author{Siyu Hu         \and
        Xuejin Chen \and
        Xin Tong
}

%\authorrunning{Short form of author list} % if too long for running head

\institute{Siyu Hu \at
	Dept. of Electronic Engineering and Information Science
	University of Science and Technology of China\\
              \email{sy891228@mail.ustc.edu.cn}           %  \\
%             \emph{Present address:} of F. Author  %  if needed
           \and
           Xuejin Chen \at
           Dept. of Electronic Engineering and Information Science
           University of Science and Technology of China\\
           \email{xjchen99@ustc.edu.cn}
           \and
           Xin Tong \at
           Microsoft Research Asia\\
           \email{xtong@microsoft.com}
}

\date{Received: date / Accepted: date}
% The correct dates will be entered by the editor


%\maketitle






%\begin{acknowledgements}
%If you'd like to thank anyone, place your comments here
%and remove the percent signs.
%\end{acknowledgements}

% BibTeX users please use one of
%\bibliographystyle{spbasic}      % basic style, author-year citations
%\bibliographystyle{spmpsci}      % mathematics and physical sciences
%\bibliographystyle{spphys}       % APS-like style for physics
%\bibliographystyle{IEEEtran}
%\bibliography{JRCS}   % name your BibTeX data base

% Non-BibTeX users please use
%\begin{thebibliography}{}
%
% and use \bibitem to create references. Consult the Instructions
% for authors for reference list style.
%
%\bibitem{RefJ}
% Format for Journal Reference
%Author, Article title, Journal, Volume, page numbers (year)
% Format for books
%\bibitem{RefB}
%Author, Book title, page numbers. Publisher, place (year)
% etc
%\end{thebibliography}
\noindent \textbf{Siyu Hu} is a Ph.D candidate in Department. of Electronic Engineering and Information Science,
University of Science and Technology of China. His research interests focus on capturing, generation, processing and meshing of 3D point cloud. Siyu obtained his B.S from Hunan University in 2013. He started his Ph.D in University of Science and Technology of China in 2013.\\

\noindent \textbf{Xuejin Chen} Xuejin Chen is an associate professor in Dept. of Electronic Engineering and Information Science in University of Science and Technology of China. Her research interests include 3D architectural modeling, geometry processing and visual recognition. She received her BSc degree in 2003 and the PhD degree in 2008 from the University of Science and Technology of China. From 2008 to 2010, she conducted research as a postdoctoral scholar in the Department of Computer Science at Yale University.  \\

\noindent \textbf{Xin Tong} is a principal research manager in internet graphics group of Microsoft Research Asia. His research interests cover variant topics in computer graphics and computer vision, including appearance modeling and rendering, texture synthesis, light transport analysis and capturing, realistic rendering, facial performance capturing, and data driven geometric processing. Xin obtained his Ph.D. from Tsinghua University in 1999. He has published more than 80 peer-reviewed papers in top graphics and vision conferences and journals.
\end{document}
% end of file template.tex

