\section{Algorithms and Implementation}
\label{sec:imp}
\subsection{Expectation Conditional Maximization}
Assuming the observed point clouds $\{V_m\}$ are independent and identically distributed, we can then write the (\ref{equ:obj0}) as: 
\begin{equation}
\label{equ:obj1}
\varepsilon(\Theta|V,Z) =\sum_{m,i,k}\alpha_{mik}(\log p_k +\log P(\phi_{nm}(v_{mi})|z_{ji}=k;\Theta))
\end{equation}
In which the $\alpha_{mik}=P(z_{mi}=k|v_{mi};\Theta)$,
\begin{algorithm}[htb]
	\caption{Joint Registration and Co-segmentation (JRCS)}
	\label{alg:jrcs}
	\textbf{Input:}~~\\
	$\{V_m\}$:Observed point clouds\\
	$\{\alpha_{mik}^0\}$:Initial posterior probabilities~~\\
	\textbf{Output:}~~\\
	$\Theta^q$:Final parameter set~~
	\begin{enumerate}
		\item $q\leftarrow0$
		\item \textbf{repeat}
		\item CM-step-a: Use $\alpha^q_{mik}$, $x^{q-1}_k$ to estimate $\{R_{mn}^q\}$ and $\{t_{mn}^q\}$
		\item CM-step-b: Use $\alpha^q_{mik}$, $\{R_{mn}^q\}$ and $\{t_{mn}^q\}$ to estimate the Gaussian centers $x_k^q$
		\item CM-step-c: Use $\alpha^q_{mik}$, $\{R_{mn}^q\}$ and $\{t_{mn}^q\}$ to estimate the covariances $\Sigma_k^q$
		\item CM-step-d: Use $\alpha^q_{mik}$ to estimate the priors $p_k^q$
		\item E-step: Use $\Theta^{q-1}$ to estimate posterior probabilities. $\alpha^q_{mik}=P(z_{mi}|v_{mi};\Theta^{q-1})$
		\item $q \leftarrow q+1$
		\item \textbf{until} Convergence
		\item \textbf{return} $\Theta^q$
	\end{enumerate}
\end{algorithm}
\subsection{Initialization Techniques}
A key advantage motivates our formulation is that the soft correspondence can be initialized more flexiblely comparing to the typical initialization techniques such as landmark point pairs in registration.\\
%\textbf{Initial Segment based on Planar Fitting}\\
%\textbf{Block Based Feature Extraction and Clustering}\\
The result of Clustering:
$$P(B_{mj} \in C_n)$$
\textbf{Soft Correspondence Initialization}\\
Then the $\alpha$ is initialized as:
$$\alpha_{ijk}=P(B_{mj} \in C_n)$$
on the condition that:
$$v_{ij} \in B_{mj} \wedge x_k \in O_n$$