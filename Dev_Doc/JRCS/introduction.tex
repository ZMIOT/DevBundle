\section{Introduction}
\label{sec:intro}
In many researches and applications of indoor scenes the data of segmented and even annotated 3D indoor scenes are required as either data base or training data (e.g.\cite{SceneFromExample}\cite{Fisher:2012:ESO:2366145.2366154}\cite{Chen:2014:ASM:2661229.2661239}\cite{Fisher:ActivityCentricSceneSynthesis}).\\
One way to build such data base is to interactively compose scenes from 3D shape models resulting in scenes with object segmentation and annotation naturally available, or to mannually segment and annotate existing scenes. This procedure can be tedious and time consuming, despite the efforts to improve the interaction experience(e.g.\cite{Merrell:2011:IFL:2010324.1964982}\cite{Xu:2013:SSC:2461912.2461968}).\\
Another way is to automatically generate scenes from 3D shape models according to the input RGB or RGB-D images(e.g.\cite{Liu2015Model}\cite{Chen:2014:ASM:2661229.2661239}). In such methods, a retrievial procedure is usually needed and inevitablely limit the result to a certain set of 3D models despite the actual 3D model in the input images.\\
We prefer a approach that helps us build such data set directly from the captured data. One of the major gap between the required data set and available scene capturing framework(e.g.\cite{KinectFusion}) is the general object level segmentation. We want to stress that a general object level segmentation problem should not be treated as an equivalence of multilabel classification problem since it is not limited to a certain set of objects. For 3D data, \cite{3DReasoningfromBlockstoStability} used some simplified physical prior knowledge (i.e. the block based stability) to help acheiving the general object segmentation, while the work of \cite{Xu:2015:ACS:2816795.2818075} proposes a practical and rather complete framework to close the gap between the required data set and available scene capturing method. One of the observation in \cite{Xu:2015:ACS:2816795.2818075} is that the motion consistency of rigid object can serve as a strong evidence of general objectness. To exploit this fact, they employ a robot to do proactive push and use the movement tracking to verify and iteratively improve their object level segmentation result. Our work presented in this paper is trying to exploit the same observation from a different approach.\\
We intend to use the motion consistency that is natrually revealed by human activities along the time. Down to this approach, we are facing the choice of scanning scheme. One way is to record the change of the scene along with the human activities, another is to arrange a daily or even a once every half day sweep to only record the result of human activities but avoid the instant of human motion. The main challenge brought in by the second scheme is that we may not be able to solve the object correspondence by a local search due to the sparse sampling over time, but the very same challenge exists in the first scheme due to the exclusion caused by human bodies not to mention other additional process(e.g. tracking and segment) needed for human bodies. With the second scanning scheme, our original intention of building 3D scene data set from capturing naturally leads us to the problem of coupled joint registration and co-segmentation. In this problem,    
      